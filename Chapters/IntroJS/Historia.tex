
\section{Historia de JavaScript}

JavaScript, comunmente abreviado como \textsc{JS}, es un lenguaje de programación interpretado. En los comienzos, el lenguaje se utilizaba para agregar dinamismo del lado del cliente a las páginas web. Sin embargo, hoy en día se pueden crear aplicaciones de escritorio o del lado del servidor.

El lenguaje fue creado por \textbf{Brendan Eich} en 1995, quien en ese entonces trabajaba para Netscape. Eich denominó a su lenguaje \texttt{LiveScript}, y el objetivo inicial del lenguaje era solucionar problemas de validación de formularios complejos en el lado del cliente para el navegador Netscape Navigator, tratando de adaptarlo a tecnologías ya existentes. 

La empresa Netscape junto con Sun Microsystems desarrollaron en conjunto este lenguaje de programación. Pero por cuestiones de mercado antes del lanzamiento, Netscape decidió cambiar el nombre del lenguaje a JavaScript (ya que en ese entonces Java estaba de moda en el mundo informático).

Al poco tiempo, la empresa Microsoft lanzó \texttt{JScript} para Internet Explorer. Para no entrar en una guerra informática, Netscape decidió que lo mejor sería estandarizar el lenguaje. Para ello, enviaron la especificación de JavaScript 1.1 al organismo ECMA (European Computer Manufacturers Association).

ECMA creó el comité TC39 con el objetivo de "\emph{estandarizar de un lenguaje de script multiplataforma e independiente de cualquier empresa}". El primer estándar que creó el comité TC39 se denominó ECMA-262, en el que se definió por primera vez el lenguaje ECMAScript (abreviado comunmente como \texttt{ES}).

Es así entonces, que cuando hablamos de JavaScript, estamos haciendo referencia a una implementación de lo que se conoce como \texttt{ECMAScript}. El estándar ha ido evolucionando con el paso del tiempo. En la actualidad, la mayoría de los navegadores corren algún intérprete que soporta la mayoría de las características de las versiones 5.1 y 6. 

Aunque la última versión sea la de ECMAScript 8 (lanzada en Junio de 2017), la versión 6 es más popular, ya que en ésta se han agregado muchos cambios significativos para el lenguaje. En este documento se hará énfasis en las versiones 5.1 y 6.
