
\subsection{Closures}
\label{sec:closures}

El término "`closure"', también conocido en español como "`clausura"' o "`cerradura"' es utilizado en varios lenguajes. Da la capacidad a las funciones a acceder y manipular variables que son externas a la función, siempre y cuando éstas estén definidas dentro del scope donde la función fue definida. Veamos un ejemplo simple:

\begin{lstlisting}[title={Creando un closure simple}]
var num = 2;
function imprimir() {
	console.log(num);
	num = 3;
}

imprimir();					// 2
console.log(num);		// 3
\end{lstlisting}

Para este caso, la variable \code{num} y la función \code{imprimir} fueron definidas en el scope global. En la línea 3, la función puede entender al identificador \code{num} y ver su valor, e incluso en la línea 4, cambia su valor. Pero la fortaleza del closure no está ahí. Si nos limitásemos únicamente a hablar de visibilidad, deberíamos estar hablando de scope. La fortaleza del closure es que "`encierra"' a las variables que están dentro del scope de la función, y las mantendrá "`vivas"' siempre que la función exista. Veámoslo con un ejemplo más concreto:

\begin{lstlisting}[title={Analizando otro closure}]
function generar() {
  var contador = 0;

  function sumar() {
    contador++;
    console.log(contador);
  }

  return sumar;
}

var contar = generar();
contar();	// 1
contar();	// 2
contar();	// 3
\end{lstlisting}

Se puede apreciar como la función \code{generar} es invocada por única vez. Al momento de invocar a la función, se crea un nuevo contexto. Sin embargo por causa del closure, ese contexto sigue con vida ya que es alcanzado desde el contexto global (mediante la variable \code{contar}) y así, el valor de \code{contador} persiste y cambia por cada llamada que hagamos a la función \code{contar}.
