\subsection{Otras cuestiones a tener en cuenta}

\subsubsection{Identificadores}

\begin{itemize}
\item Un identificador debe \textbf{comenzar} con una letra, signo pesos (\$), o guión bajo (\textunderscore).
\item Un identificador consiste en letras, números, signo pesos (\$), o guión bajo (\textunderscore).
\item Se permiten caracteres Unicode.
\item No se permite el uso de palabras reservadas como identificadores.
\end{itemize}

\subsubsection{Sensible a las mayúsculas}

JavaScript es un lenguaje sensible a las mayúsculas, lo que significa que se entiende a \code{miVariable} y a \code{MIVARIABLE} como dos identificadores totalmente diferentes.

\subsubsection{Sin tipado estático}

El lenguaje no posee tipado estático. Sin embargo con TypeScript (de Microsoft) o Flow (de Facebook) se puede alcanzar esto mediante el uso de "`type annotations"'. Consiste en utilizar el lenguaje haciendo anotaciones de los tipos, para luego hacer un chequeo de tipos estáticos mediante un preprocesado del código. Tanto TypeScript como Flow son \textit{extensiones} de JavaScript. 
