\subsection{Tipos primitivos}
\label{subsec:introtiposprimitivos}

\subsubsection{Undefined} 
El tipo indefinido tiene un único valor, \code{undefined}. A toda variable que aún no se le haya asignado valor, tendrá el valor \code{undefined}.

\subsubsection{Null} 
El tipo nulo tiene un único valor, \code{null}, que representa al valor nulo o "`vacío"'.

\subsubsection{Boolean} 
El tipo booleano representa una entidad lógica con dos posibles valores, \code{true} ó \code{false}.

\subsubsection{String} 
Utilizado para representar datos de texto, el tipo String está definido como cero o más elementos, donde cada elemento es un entero no signado de 16 bits, de una longitud máxima de $2^{52}-1$ elementos.

\subsubsection{Number} 
Representa al conjunto de datos numérico. Se basa en la norma IEEE 754-2008, formato doble precisión de 64 bits en la aritmética de punto flotante. Toma algunos valores especiales de este conjunto para representar datos como \code{NaN} (Not a Number) y también \code{+Infinity} y \code{-Infinity}. La cantidad de valores reservados para \code{NaN} es dependiente de la implementación.

\subsubsection{Symbol} 
Fue agregado en la versión de ES6. Abarca el conjunto de todos los valores no String que pueden ser usados como clave en la propiedad de un \code{Object}. Cada valor posible de \code{Symbol} es único e inmutable. Se los puede pensar como tokens que sirven como identificadores únicos. 

\subsubsection{Object} 
Es la forma básica de representar un objeto en JavaScript. Está compuesto por una colección de propiedades.

Las propiedades se identifican usando claves. El valor de una clave puede ser un \code{String}, o \code{Symbol}. Los valores de las propiedades pueden ser de cualquiera de los tipos primitivos mencionados.
