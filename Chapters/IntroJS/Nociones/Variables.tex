\subsection{Variables}

Para la declaración de variables, el lenguaje posee la palabra reservada \code{var}. Una variable tendrá el valor inicial \code{undefined} a menos que se la inicialice en su declaración.

También se pueden hacer múltiples declaraciones en la misma línea, incluso con la asignación de un valor inicial.

\begin{lstlisting}[title={Declarando variables}]
var a;									// Definiendo una variable con nombre a
var b = 1; 	 						// Definiendo una variable con nombre b
var c, d, e;	 					// Definiendo varias variables
var f, g = true, h;			// Esto tambien es valido
var i = "Hola", j = 2;	// Definiendo y asignando multiples variables
\end{lstlisting}

Vale la pena hacer mención también a dos nuevas formas de definir variables a partir de ES6. Se trata de \code{let} y \code{const}.

Sobre \code{let}, es una forma de declarar variables de alcance local. Se hará énfasis en este punto en futuros capítulos cuando se muestren los problemas de alcance que posee el lenguaje.

Por el lado de \code{const}, se tratan de variables de valor constante, cuyo valor no se puede cambiar y tampoco pueden ser redeclaradas.
