
\subsection{Palabras reservadas}

Las palabras reservadas del lenguaje se dividen en cuatro conjuntos:

\begin{itemize}
\item Palabras claves (\textit{keywords})
\item Palabras reservadas a futuro
\item Literal nulo (\code{null})
\item Literales booleanos (\code{true} y \code{false})
\end{itemize}

Las siguientes son palabras claves, a excepción de \code{null}, \code{true} y \code{false}, que son literales.

\begin{table}[!h]
\caption{Lista de palabras claves del lenguaje.}
\label{tab:reservedkeywords}
\centering
\begin{tabular}{l l l l}
\toprule
break & do & import & throw\\
case & else & in & true\\
catch & export & instanceof & try\\
class & extends & new & typeof\\
const & false & null & var\\
continue & finally & return & void\\
debugger & for & super & while\\
default & function & switch & with\\
delete & if & this & yield\\
\bottomrule\\
\end{tabular}
\end{table}

Por otro lado, existe un conjunto de palabras reservadas a futuro. En un principio son solamente dos: \code{await} y \code{enum}. Pero si se especifica la directiva de \textit{strict mode}, aparecen otras más: \code{implements}, \code{interface}, \code{package}, \code{private}, \code{protected} y \code{public}.

En resumen, las palabras reservadas a futuro (en modo estricto) son:

\begin{table}[!h]
\caption{Lista de palabras reservadas a futuro.}
\label{tab:futurereservedkeywords}
\centering
\begin{tabular}{l l l l}
\toprule
await & implements & package & protected\\
enum & interface & private & public\\
\bottomrule\\
\end{tabular}
\end{table}
