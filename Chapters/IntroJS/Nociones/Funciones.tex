
\subsection{Funciones}
\label{sec:funciones}

Las funciones en JavaScript son objetos, instancia de \code{Function}. Al ser objetos, las mismas pueden ser guardadas como valores dentro de variables. Existen diferentes maneras de declarar una función. A continuación se enumeran 

\subsubsection{Funciones como expresión}

Una función como expresión (o función expresión) es una expresión que produce un valor, en este caso un objeto función, y luego es asignado a una variable.

\begin{lstlisting}[title={Función expresión}]
var suma = function (x, y) { return x + y };

suma(2,3); // devuelve 5
\end{lstlisting}

\subsubsection{Funciones como declaración}

Una función como declaración (o función declaración) funciona de la misma forma que la función expresión, a diferencia de que no es necesario recurrir a la asignación sino que ésta se hace automáticamente.

\begin{lstlisting}[title={Función declaración}]
function suma(x, y) { return x + y };

suma(2,3); // devuelve 5
\end{lstlisting}

\subsubsection{Funciones anónimas y arrow functions}

Se les llama funciones anónimas simplemente a aquellas funciones que no tienen nombre. Notar el detalle que cuando se habló de función como expresión, el lado derecho de la asignación era una función anónima.

Por otro lado, a partir de ES6 se introdujo el concepto de \textit{arrow function} (o \textit{fat arrow}, el cual está actualmente en varios lenguajes. Para el caso de JS, establece una forma más simple y concisa para escribir funciones anónimas, bajo el detalle de que además la palabra \code{this} dentro de la función no cambia ni está ligada a un nuevo contexto, sino que sigue está ligada léxicamente al contexto donde fue invocada. Esto último no sucede con las funciones como expresión comunes, ya que en éstas se crea un scope nuevo.

\begin{lstlisting}[title={\textit{Arrow function}}]
var mostrar = (texto) => console.log(texto);

mostrar('hola!'); // hola!
\end{lstlisting}