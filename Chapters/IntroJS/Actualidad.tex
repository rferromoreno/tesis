\section{JavaScript en la actualidad}

Después de más de 20 años de existencia, los usos del lenguaje han cambiado. JavaScript ya no es más un lenguaje para hacer validaciones de formularios complejos en páginas de Internet, ni tampoco para agregar dinamismo o animaciones a las páginas.

En la actualidad, se puede afirmar que JavaScript está en "`la cresta de la ola"'. ?`Qué se puede hacer en la actualidad con JavaScript?

\begin{itemize}
	\item Páginas Web -- Pareciera la respuesta obvia, sin embargo la forma de crear sitios web ha cambiado con el paso del tiempo. Hoy en día existe una gran cantidad de librerías y frameworks basados en JavaScript, tales como \keyword{React}, \keyword{AngularJS} o \keyword{Vue.JS}, entre otros.
	\item Aplicaciones móviles -- Se pueden crear aplicaciones para celulares o dispositivos móviles programando en JavaScript, usando \keyword{Apache Cordova}, \keyword{Sencha}, \keyword{Ionic}, \keyword{NativeScript} o \keyword{Tabris.JS}.
	\item Aplicaciones de escritorio -- Así como recién se hizo mención de las aplicaciones móviles, las de escritorio no se quedan atrás. Algunos frameworks como \keyword{Electron} ó \keyword{NW.JS} permiten crear aplicaciones multiplataforma.
	\item Robots -- Mediante frameworks como \keyword{Cylon.JS} se pueden manejar dispositivos de hardware o robots. También existen kits basados en Arduino para programar en JS, tales como \keyword{Johnny-Five} o \keyword{Nodebots}.
	\item Aplicaciones de consola -- Existen librerias que facilitan el uso de la creación de aplicaciones de linea de comandos (CLI).
	\item Machine Learning -- Así como Python tiene una gran base de librerías para prototipar sistemas que apliquen Machine Learning, en este último tiempo la comunidad de JavaScript ha seguido los mismos pasos.
\end{itemize}
