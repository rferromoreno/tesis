\subsection{Estructuras condicionales}

El lenguaje posee las estructuras condicionales, nuevamente con similitud sintáctica a los lenguajes de la familia de C. A continuación se muestran algunos ejemplos de algunas de éstas estructuras, que servirán para mejor entendimiento del código de ejemplo dado en los capítulos siguientes.

\subsubsection{Condicionales \code{if} e \code{if-else}}

\begin{lstlisting}[title={Ejemplos de \code{if} e \code{if-else}}]
if (a > 0) {
	// bloque si la condición es verdadera 
}

if (a > 0) {
	// bloque si la condición es verdadera
} else {
	// bloque si la condición es falsa
}
\end{lstlisting}

\subsubsection{Operador ternario \code{?:}}

\begin{lstlisting}[title={Operador ternario \code{?:}}]
var mayor = a > b ? a : b
\end{lstlisting}