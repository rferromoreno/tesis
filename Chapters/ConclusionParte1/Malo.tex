\section*{Lo malo}

La falta de soporte para números "`grandes"' es un punto en contra. Actualmente existen propuestas para introducir el concepto de \code{BigInt} (decimales de precisón arbitraria), sin embargo falta avanzar sobre éste campo. Mientras tanto, JavaScript es un lenguaje poco amigable con los números. En este sentido, el lenguaje se aleja un poco del dominio de las aplicaciones científicas.

El uso de \code{const} puede resultar un poco confuso para el programador inexperto. Hace que una variable no sea re-asignable, pero no significa que su valor sea inmutable. Sería deseable que el lenguaje agregara algún constructor para éste último caso, para brindar con naturaleza el concepto de inmutabilidad. En el capítulo \ref{ch:pf} se hablará más sobre éste concepto.

Por la relación entre el nombre de Java y JavaScript, además de su similitud sintáctica, es un error grave en pensar que su semántica es la misma. Si bien gran parte de éste malentendido puede ser responsabilidad del programador, hay que ser criteriosos también con el lenguaje, el cual adoptó bastantes aspectos sintácticos de la familia de lenguajes de C.

Siguiendo con el sentido semántico, el concepto de \code{this} puede resultar difícil de comprender para el programador que viene de lenguajes con herencia clásica. Agregar la tarea al programador de tener conocimiento en qué contexto una función es ejecutada significa una sobrecarga para quien use el lenguaje.