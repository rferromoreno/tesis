\section{Lo bueno}

El sistema de tipos de JavaScript lo hace un lenguaje excesivamente flexible. La falta de burocracia a la hora de especificar tipos es lo que lo hace un verdadero lenguaje de scripting. Por otro lado, la similitud sintáctica con la familia de lenguajes de C hacen que de cierta forma sea fácil de aprender. 

Algo destacable es la \textit{unicidad} en cuanto a sus entidades. \textit{Casi todo es un objeto}. Esta característica lo hace excepcional y bastante poderoso. La habilidad de poder pasar funciones, arreglos, clases y módulos como parámetros en una función es una característica que no todos los lenguajes poseen.

El uso de funciones para creación de un scope local es sin dudas un acierto, pero mayor cierto fue la introducción de scope por bloques con el uso de \code{const} y \code{let}. Dicha característica ayuda a mantener limpio el espacio de nombres, y declarar variables de forma léxica.