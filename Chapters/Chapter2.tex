% Chapter 2

\chapter{Sistema de Tipos} % Main chapter title

\label{Chapter2} % For referencing the chapter elsewhere, use \ref{Chapter1} 

%----------------------------------------------------------------------------------------

JavaScript es un lenguaje de "`scripting"', con tipado dinámico y un sistema de tipos débil. Generalmente, en este tipo de lenguajes interpretados no es necesario definir el tipo de una variable al momento de declararla, por lo que es lógico preguntarse ?`Es JavaScript un lenguaje seguro?.

%----------------------------------------------------------------------------------------

\section{Tipos primitivos}

Tal como se mencionó en el Capítulo \ref{Chapter1}, existen solamente 7 tipos primitivos del lenguaje:

\begin{itemize}
\item \code{undefined}
\item \code{null}
\item \code{number}
\item \code{string}
\item \code{boolean}
\item \code{symbol}
\item \code{object}
\end{itemize}

Algunos autores consideran que \code{Object} no es un tipo primitivo, sino que es un tipo que hereda de \code{Null}. Por otro lado, hay que mencionar que las funciones en JavaScript son consideradas objetos, por lo que \code{Function} es un subtipo de \code{Object}.

%----------------------------------------------------------------------------------------
