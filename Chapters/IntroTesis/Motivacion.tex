\section{Motivación}

Hay una realidad de la que no se puede escapar: Hoy por hoy, JavaScript es uno de los lenguajes más utilizados a nivel empresarial. No solamente eso, sino que gracias a la aparición de Node en 2009, junto con una gran actualización del estándar ECMAScript en 2015 (también conocido como ES6), el lenguaje fue comenzó a ser tratado de una forma más seria, abriéndose a nuevos dominios de aplicaciones donde antes no era considerado como posible solución.

Otro detalle, es que JavaScript "`ganó"' terreno en todo lo relativo a aplicaciones web, por haber sido de alguna forma un lenguaje pionero en este aspecto. Además de que tanto su comunidad, los frameworks y las librerías disponibles en éste lenguaje se han incrementado en cantidades exponenciales. A nivel \textit{frontend}, en la actualidad los frameworks más populares usados son React, Angular y Vue, mientras que jQuery se quedó un poco atrás pero sigue en vigencia. Todos ellos están basados en JavaScript, lo que significa que conocerlos en detalle, implica conocer JavaScript. Por el lado del \textit{backend}, Node también está haciendo lo suyo. No solo eso, sino que además gracias a la potencia del intérprete V8, se está expandiendo por otros dominios de aplicación, como pueden ser las aplicaciones de escritorio o la robótica.

Sin embargo, es imposible evitar leer críticas del lenguaje, a veces infundadas. La gama de opiniones y críticas es muy amplia, yendo desde el fanatismo por el lenguaje hasta el odio más profundo. Siendo un lenguaje muy criticado pero a su vez también muy usado, entonces ?`quién se equivoca? ?`quién dice la verdad? ?`quién tiene la razón?. Para poder llegar a entender sobre éstos aspectos, primero es necesario entender qué pasa a bajo nivel. Cómo está diseñado el lenguaje, cómo se comporta e incluso porqué.

La motivación de hacer una tesis analizando el lenguaje está atada a los siguientes objetivos:

\begin{itemize}
\item Entender características profundas del lenguaje para dominarlo en gran parte.
\item Ganar experiencia en el área. Poder analizar, trazar y debuggear código con facilidad.
\item Corroborar o desmitificar las críticas que se le suelen hacer al lenguaje.
\item Analizar la relación del lenguaje con respecto a los paradigmas de programación populares.
\end{itemize}