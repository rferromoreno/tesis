\chapter{Introducción a la tesis} % Main chapter title

\label{ch:introtesis} % For referencing the chapter elsewhere, use \ref{Chapter1} 

\section{Motivación}

Hay una realidad de la que no se puede escapar, que es que hoy Javascript es uno de los lenguajes más utilizados a nivel empresarial. No solamente eso, sino que gracias a la aparición de Node en 2009, junto con una gran actualización del estandar ECMAScript en 2015 (también conocido como ES6), el lenguaje fue empezado a ser tratado de una forma más seria, abriéndose a nuevos dominios de aplicaciones, donde antes Javascript no era considerado como una posible solución.

Otro detalle, es que Javascript "`ganó"' demasiado terreno siendo el lenguaje pionero para ser implementado en los navegadores, además de que su comunidad ha crecido a un ritmo enorme en los últimos años. A nivel \textit{frontend} en la web, al día de hoy los frameworks más populares usados son React, Angular y Vue, mientras que jQuery se quedó un poco atrás pero sigue en vigencia. Esos cuatro frameworks utilizan Javascript. Por el lado del \textit{backend}, Node también está haciendo lo suyo. No solo eso, sino que además lo expande a otros dominios de aplicación, como pueden ser las aplicaciones de escritorio o la robótica.

Sin embargo es imposible evitar leer críticas del lenguaje, a veces infundadas. Agregando el dato de que la gama de opiniones y críticas es muy amplia, yendo desde el fanatismo del lenguaje hasta el odio profundo. Dicho esto, la motivación de hacer una tesis analizando el lenguaje está atada a los siguientes objetivos:

\begin{itemize}
\item Entender características profundas del lenguaje para comprender qué sucede a bajo nivel.
\item Ganar experiencia en el área. Poder analizar, trazar y debuggear código con facilidad.
\item Corroborar o desmitificar las críticas que se le suelen hacer al lenguaje.
\item Analizar la relación del lenguaje con respecto a los paradigmas de programación populares.
\end{itemize}

\section{Estructura de la tesis}

El documento tiene un capítulo de introducción al lenguaje donde se presentará tanto su historia, como también características subyacentes y conceptos que serán de uso para entender los capítulos siguientes. Luego, la tesis se divide en dos partes: "`Sistema de tipos"' y "`Paradigmas de programación"'. 

En la primera parte se hablará sobre características, debilidades y fortalezas dentro del sistema de tipos del lenguaje. Se hará mención sobre los tipos que tiene el lenguaje y algunas incoherencias sobre los mismos. También se revisarán cuestiones sobre coerción en algunas expresiones, y parte de cómo trabaja el scope del lenguaje.

La segunda parte trata sobre Javascript y los paradigmas de programación, donde se lo evalúa frente al paradigma funcional y el orientado a objetos. Se intentará determinar cuán cercano es el soporte que tiene el lenguaje ante las características de éstos paradigmas.

Cada parte tiene una sección de conclusiones, detallando \textit{lo bueno, lo malo y lo feo} de los temas tratados. No se intenta imponer una opinión como una única verdad, sino más bien acercar al lector mostrando fortalezas y debilidades del lenguaje, para que éste pueda sacar sus propias conclusiones.

La tesis se resume con una conclusión general sobre lo visto, y una breve recopilación sobre conceptos que no han sido tratados en el documento, pero destacables en caso de que el lector quisiera seguir ampliando la investigaciónn.