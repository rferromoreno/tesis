
\section{Estructura de la tesis}

El documento tiene un capítulo de introducción al lenguaje donde se presentará tanto su historia, como también características subyacentes y conceptos que serán de utilidad para entender los capítulos siguientes. Luego, la tesis se divide en dos partes: "`Sistema de tipos"' y "`Paradigmas de programación"'. 

En la primera parte veremos las características, debilidades y fortalezas dentro del sistema de tipos del lenguaje. Se hace mención sobre los tipos que tiene el lenguaje y algunas incoherencias en su uso. También se revisarán aspectos sobre coerción en algunas expresiones, y parte de cómo trabaja el scope del lenguaje.

La segunda parte trata sobre JavaScript y los paradigmas de programación, donde se lo evalúa frente al paradigma funcional y el orientado a objetos. Se intentará determinar cuán cercano es el soporte que tiene el lenguaje ante las características esperadas de éstos paradigmas.

Cada parte tiene una sección de conclusiones, detallando \textit{lo bueno, lo malo y lo feo} de los temas tratados. No se intenta imponer una opinión como una única verdad, sino más bien acercar al lector mostrando fortalezas y debilidades del lenguaje, para que éste pueda sacar luego sus propias conclusiones.

La tesis se resume con una conclusión general sobre lo visto, y una breve recopilación sobre conceptos que no han sido tratados en el documento, pero destacables en caso de que el lector quisiera seguir ampliando la investigación.