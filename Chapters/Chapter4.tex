\chapter{Paradigma orientado a objetos}

\label{Chapter4}

% -----

Uno de los debates principales sobre JavaScript es su soporte hacia el paradigma orientado a objetos.

% -----

\section{Clases}


Previo a la salida de ES6 (es decir, en la version 5 de JavaScript) la creación de clases en JavaScript se realiza mediante el uso de patrones para la creación de objetos. La realidad es que en JavaScript no existe un soporte formal o natural para las clases, sino que hay que recurrir a estos patrones para simularlo. Éstos son:

\begin{itemize}
	\item Factory
	\item Constructor
	\item Prototype
	\item Dynamic Prototype
\end{itemize}

\section{Herencia}

Como se ha mencionado anteriormente, JavaScript tiene la particularidad de tener la herencia prototipada en vez de la herencia clásica.



\section{Encapsulamiento}

\section{Polimorfismo}

\section{Abstracción}

\section{Modularidad}

\section{Principio de ocultación}

\section{Lo bueno, lo malo, lo feo...}