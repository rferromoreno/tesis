\section{Conversión explícita}
\label{sec:conversionexplicita}

Comenzaremos este capítulo hablando de la conversión explícita, para entender qué valores de un tipo corresponden a otro tipo. Existen cuatro funciones para hacer conversión a tipos built-in del lenguaje, las cuales son las funciones \code{Boolean}, \code{Number}, \code{String} y \code{Object}. Es innecesario pensar funciones para convertir a \code{null} y \code{undefined} dado que son casos especiales. 

\subsection{Boolean} 

La función \code{Boolean()} convierte el argumento dado a un valor booleano. Los valores que se detallan a continuación se convierten a \code{false}, y son llamados valores de falsedad ("`\textit{falsy}"').

\begin{itemize}
\item \code{undefined}
\item \code{null}
\item \code{false}
\item \code{0}
\item \code{NaN}
\item \code{''}
\end{itemize}

El resto de los valores son considerados valores de verdad ("`\textit{truthy}"') y serán convertidos a \code{true}, incluyendo a los objetos, que todos son convertidos a true.

\subsection{Number}

Análogamente, la función \code{Number()} convierte un valor dado a su representación numérica. Para éste caso, las reglas son un poco más complejas (y poco intuitivas)

\begin{itemize}
\item \code{undefined} se transforma en \code{NaN}.
\item \code{null} se transforma en \code{0}.
\item \code{false} se transforma en \code{0}.
\item \code{true} se transforma en \code{1}.
\item Para el caso de \code{string}, se hace \textit{parsing} de la siguiente manera: Se remueven los espacios en blanco, si el \code{string} resultante es \code{''}, se transforma en \code{0}, sino se intentará leer el valor numérico (si posee algún caracter ajeno a un valor numérico, se transformará en \code{NaN}, caso contrario, se transforma en dicho valor numérico).
\item Para el caso de \code{object}, primero se transforma a primitivo (ver sección \ref{sec:toprimitive}), y luego es convertido a \code{number} con las reglas recién mencionadas.
\end{itemize}

\subsection{String}

La función \code{String()} es el caso más obvio para la mayoría de los tipos primitivos. Convertirá el valor dado al literal string del tipo dado.

\begin{itemize}
\item \code{null} se convierte a \code{"null"}.
\item \code{undefined} se convierte a \code{"undefined"}.
\item \code{true} y \code{false} se convierten a \code{"true"} y \code{"false"}, respectivamente.
\item Los valores numéricos se transforman en su \code{string} equivalente. Por ejemplo, \code{123.45} se transformará en \code{"123.45"}.
\item Para el caso de \code{object}, primero se transforma a primitivo (ver sección \ref{sec:toprimitive}), y luego es convertido a \code{string} con las reglas recién mencionadas.
\end{itemize}

\subsection{Object}

Para el caso de \code{Object()} es un poco más compejo de reflejar.

\begin{itemize}
\item \code{null} se convierte en el objeto vacío \code{\{\}}.
\item \code{undefined} se convierte en el objeto vacío \code{\{\}}.
\item Para los casos de \code{string}, \code{number} y \code{boolean}, las primitivas se convierten en "`primitivas envueltas"'. Esto es, "`objetos"' que son instancias de \code{String}, \code{Number} y \code{Boolean}, respectivamente.
\item Los objetos se vuelven en sí mismos.
\end{itemize}