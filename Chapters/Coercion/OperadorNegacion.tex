\section{El operador \code{!}}
\label{sec:operadornegacion}

El operador unario de negación lógica es otro de los operadores que es necesario hacer mención. Tal como sucedía con el operador unario \code{+}, en donde se hacía una conversión a Number, con el operador \code{!} sucede análogamente lo mismo, pero la conversión será a Boolean. 

\begin{lstlisting}[title={Operador \code{!} con diferentes valores}]
!true					// false
!false				// true
!null					// true
!undefined		// true
!0						// true
!1						// false
!{}						// false
![]						// false
!function(){}	// false
\end{lstlisting}

Haciendo uso de este conocimiento sobre la conversión a \code{boolean}, podemos llegar a tener código más compacto. Por ejemplo, si tuvieramos una variable \code{a} y necesitamos chequear en algún momento si \code{a === null} o sí \code{a === undefined} para saber si \code{a} obtuvo algún valor, se puede hacer de la siguiente manera:

\begin{lstlisting}
var a;
// ...
if (!a) {
	// lanzar un error
}
\end{lstlisting}

De todas formas, hay que tener especial cuidado con esto, ya que por ejemplo si nuestra variable \code{a} obtuviera el valor \code{0}, nuestro código lanzaría un error cuando en realidad sí se obtuvo un valor.