\chapter{Coerción} % Main chapter title

\label{ch:coercion} % For referencing the chapter elsewhere, use \ref{Chapter1} 

Cuando un lenguaje tiene seguridad en su sistema de tipos, hace que el mismo sea coherente, predecible, que sea fácil de intuir los tipos que se manejan. Para que un lenguaje gane flexibilidad en su sistema de tipos, debe sacrificar éstos aspectos de seguridad. Sin embargo, se pierden también los calificativos recién mencionados.

Una de las mayores críticas a Javascript es sobre la conversión de tipos implícita que ocurre en las expresiones. Tiene la fama de ser un lenguaje extremadamente flexible, pero al mismo tiempo impredecible en cuanto a la coerción. En este capítulo se hablará de los efectos de la conversión de tipos que ocurre en el lenguaje, tanto la implícita como la explícita, y así podremos acercarnos a una conclusión de qué tan relajado es el sistema de tipos con respecto a la coerción.

\section{Conversión explícita}
\label{sec:conversionexplicita}

Comenzaremos este capítulo hablando de la conversión explícita, para entender qué valores de un tipo corresponden a otro tipo. Existen cuatro funciones para hacer conversión a tipos built-in del lenguaje, las cuales son las funciones \code{Boolean}, \code{Number}, \code{String} y \code{Object}. Es innecesario pensar funciones para convertir a \code{null} y \code{undefined} dado que son casos especiales. 

\subsection{Boolean} 

La función \code{Boolean()} convierte el argumento dado a un valor booleano. Los valores que se detallan a continuación se convierten a \code{false}, y son llamados valores de falsedad ("`\textit{falsy}"').

\begin{itemize}
\item \code{undefined}
\item \code{null}
\item \code{false}
\item \code{0}
\item \code{NaN}
\item \code{''}
\end{itemize}

El resto de los valores son considerados valores de verdad ("`\textit{truthy}"') y serán convertidos a \code{true}, incluyendo a los objetos, que todos son convertidos a true.

\subsection{Number}

Análogamente, la función \code{Number()} convierte un valor dado a su representación numérica. Para éste caso, las reglas son un poco más complejas (y poco intuitivas)

\begin{itemize}
\item \code{undefined} se transforma en \code{NaN}.
\item \code{null} se transforma en \code{0}.
\item \code{false} se transforma en \code{0}.
\item \code{true} se transforma en \code{1}.
\item Para el caso de \code{string}, se hace \textit{parsing} de la siguiente manera: Se remueven los espacios en blanco, si el \code{string} resultante es \code{''}, se transforma en \code{0}, sino se intentará leer el valor numérico (si posee algún caracter ajeno a un valor numérico, se transformará en \code{NaN}, caso contrario, se transforma en dicho valor numérico).
\item Para el caso de \code{object}, primero se transforma a primitivo (ver sección \ref{sec:toprimitive}), y luego es convertido a \code{number} con las reglas recién mencionadas.
\end{itemize}

\subsection{String}

La función \code{String()} es el caso más obvio para la mayoría de los tipos primitivos. Convertirá el valor dado al literal string del tipo dado.

\begin{itemize}
\item \code{null} se convierte a \code{"null"}.
\item \code{undefined} se convierte a \code{"undefined"}.
\item \code{true} y \code{false} se convierten a \code{"true"} y \code{"false"}, respectivamente.
\item Los valores numéricos se transforman en su \code{string} equivalente. Por ejemplo, \code{123.45} se transformará en \code{"123.45"}.
\item Para el caso de \code{object}, primero se transforma a primitivo (ver sección \ref{sec:toprimitive}), y luego es convertido a \code{string} con las reglas recién mencionadas.
\end{itemize}

\subsection{Object}

Para el caso de \code{Object()} es un poco más compejo de reflejar.

\begin{itemize}
\item \code{null} se convierte en el objeto vacío \code{\{\}}.
\item \code{undefined} se convierte en el objeto vacío \code{\{\}}.
\item Para los casos de \code{string}, \code{number} y \code{boolean}, las primitivas se convierten en "`primitivas envueltas"'. Esto es, "`objetos"' que son instancias de \code{String}, \code{Number} y \code{Boolean}, respectivamente.
\item Los objetos se vuelven en sí mismos.
\end{itemize}

\section{ToPrimitive}
\label{sec:toprimitive}

La función \code{ToPrimitive} está especificada en el estandar de ECMAScript y es un pilar fundamental para entender las secciones siguientes. En la sección anterior (\ref{sec:conversionexplicita}) se hizo mención sobre la conversión de \code{object} hacia otros tipos. Para el caso de boolean, es simple dado que todos los objetos se convierten a \code{true}. Sin embargo, para los tipos \code{number} y \code{string} es un poco más complejo, dado que hay que recurrir a este algoritmo. 

?`Por qué un objeto se querría convertir a un tipo primitivo? Es justamente una de las partes flexibles del lenguaje. Esta función es la que permite poder utilizar operandos de distintos tipos en algunas expresiones. Un objeto se podría tener que convertir a un tipo numérico para el caso de algunas operaciones matemáticas (por ejemplo, el tipo \code{Date} donde se pueden restar fechas) o por ejemplo a un texto, para el caso de que se necesite mostrar un objeto (ya sea mediante las funciones \code{alert} o \code{console.log}).

Cabe aclarar que la función \code{ToPrimitive} existe en la especificación de ECMAScript, pero no es un método o una función tal como se mostró con \code{Boolean}, \code{Number}, \code{String} y \code{Object}. Este método de conversión se utilizará siempre que el contexto lo requiera así, generalmente en operaciones donde ocurre la coerción.

La signatura del método es la siguiente: \code{\color{blue}ToPrimitive(input, PreferredType?)}

El parámetro opcional \code{PreferredType} representa la preferencia del tipo a convertir, y el mismo puede ser \code{Number} o \code{String}. Por lo general, se asume que cuando no está presente, el valor por defecto es \code{Number}, sin embargo este comportamiento puede ser cambiando sobreescribiendo el método \code{@@toPrimitive}, como sucede en el caso de \code{Date} y \code{Symbol}.

El algoritmo de \code{ToPrimitive} si el \code{PreferredType} es \code{Number} es el siguiente:

\begin{enumerate}
\item Si \code{input} es de tipo primitivo, retornarlo.
\item Sino, si \code{input} es un objeto, llamar a \code{input.valueOf()}. Si el resultado es primitivo, retornarlo.
\item Sino, llamar a \code{input.toString()}, si el resultado es primitivo, retornarlo.
\item Sino, lanzar un \code{TypeError}.
\end{enumerate}

En caso de que el \code{PreferredType} sea tipo \code{String}, el algoritmo es análogo, solo que se cambian de lugares los puntos 2 y 3.

\section{El operador \code{+}}
\label{sec:operadormas}

Uno de los grandes dilemas a la hora de diseñar un lenguaje es sobre el uso de operadores como el \code{+}. ?`Debería este operador estar sobrecargado?. Es simple pensar en el \code{+} como un operador de suma de dos datos numéricos, pero ?`se podría usar para concatenar dos cadenas de texto? ?`o para unir el contenido de dos listas (arreglos)?.

\subsection{Operador unario}

La existencia del operador unario \code{+} puede resultar en una de las más confusas del lenguaje. La única finalidad del operador unario \code{+} es hacer una conversión numérica en caso de ser posible, y en caso contrario retornar \code{NaN}. No funciona de la misma forma en la que funcionaría el operador unario \code{-} (negación matemática), en donde además de realizar la conversión numérica, intentará cambiar el signo. Por ejemplo:

\begin{lstlisting}[title={Operador unario \code{+}}]
+3				// 3
-3				// -3
+(-3)			// -3
+-3				// -3
-+3				// -3
\end{lstlisting}

Nuevamente, este operador funciona únicamente para una conversión a número, pero no del signo. Dicho esto, se presentan a continaución algunos ejemplos con sus respectivos resultados.

\begin{lstlisting}[title={Operador unario \code{+} (más casos)}]
+3                                   // 3
+'-3'                                // -3
+'3.14'                              // 3.14
+'3'                                 // 3
+'0xFF'                              // 255
+true                                // 1
+'123e-5'                            // 0.00123
+false                               // 0
+null                                // 0
+'Infinity'                          // Infinity
+'infinity'                          // NaN
+function(val){  return val }        // NaN
\end{lstlisting}

El operador unario \code{+} suele ser el elegido por algunos programadores cuando quieren hacer coerción explícita de un valor a su tipo numérico. Los resultados de utilizar \code{Number()} para ambos casos serían idénticos. El \textit{trade-off} está en la legibilidad de las expresiones cuando éstas se vuelven más complejas.

\subsection{Operador binario}

Para comenzar, iremos con el ejemplo más simple de todos, el obvio: El operador \code{+} sirve como suma a la hora de trabajar con datos numéricos.

\begin{lstlisting}[title={Operador \code{+} en números}]
1 + 2 			// 3
10 + 50 		// 60
.1 + 0			// 0.1
NaN + NaN		// NaN
-1 + 10			// 9
\end{lstlisting}

Por otro lado, el mismo operador también se puede usar para concatenación de strings.

\begin{lstlisting}[title={Operador \code{+} en strings}]
"hola" + "mundo"	// "holamundo"
"abc" + "def"			// "abcdef"
"123" + "456"			// "123456"
"chau" + ""				// "chau"
\end{lstlisting}

Hasta este punto, todo parece funcionar de acuerdo a lo esperado. El problema es cuando empezamos a mezclar los tipos. Empecemos mirando los casos de \code{number} y \code{string}:

\begin{lstlisting}[title={Operador \code{+} mezclando strings con números}]
1 + "1"				// "11"
"1" + 1				// "11"
2 + ""				// "2"
"hola" + 0		// "hola0"
\end{lstlisting}

Si bien puede parecer un poco raro, la regla aquí es que si uno de los operandos es \code{string}, entonces el resultado será la concatenación de los \code{string}. Ahora sigamos evaluando, qué pasa con los \code{boolean}:

\begin{lstlisting}[title={Operador \code{+} en booleanos}]
true + true		// 2
true + false	// 1
false + false	// 0
\end{lstlisting}

Tal como se mencionó en la sección \ref{sec:conversionexplicita}, el valor numérico de \code{true} es \code{1} mientras que el de \code{false} es \code{0}. Entonces para este caso, ocurre una coerción hacia tipo numérico en los operandos, y el operador \code{+} funciona como una suma numérica normal.

Para entender mejor lo que está sucediendo, es preferible tener a mano un pseudo algoritmo de lo que sucede por detrás con el operador \code{+}:

\begin{enumerate}
\item Convertir ambos operandos a sus valores primitivos haciendo uso del \code{ToPrimitive} mencionado anteriormente.
\item Si alguno de los dos operandos es \code{string}, convertir ambos operandos a \code{string} y retornar la concatenación de los resultados.
\item Sino, convertir ambos operandos a su valor numérico y retornar la suma de los resultados.
\end{enumerate}

Considerando esto, evaluemos algunos casos más:

\begin{lstlisting}[title={Operador \code{+} con arreglos y objetos}]
[] + []			// ''
[] + {}			// '[object Object]'
{} + {}			// '[object Object][object Object]'
{} + []			// 0
\end{lstlisting}

Para el primer caso, la conversión a primitivo del arreglo vacío \code{[]} primero hace un \code{valueOf()}, el cual retorna el arreglo mismo (\code{this}). Luego, como el resultado no es un valor primitivo, se hace \code{toString()} el cual retorna el string vacío \code{''}. Finalmente, el resultado de sumar \code{[] + []} es el string vacío.

Para el segundo caso, sucede algo similar. Dado que el primer operando es el string vacío, entonces el \code{+} corresponde a una concatenación de strings. Y el método \code{toString()} sobre un objeto (instancia directa de Object) se traduce a string de la forma \code{'[object Object]'}.

Luego de haber visto los dos primeros casos, el tercer caso se explica solo. Se hace la conversión del primero y el \code{+} representa la concatenación de strings. Sucede exactamente lo mismo que en el primer caso, el \code{valueOf()} del primer operando retorna \code{this} entonces se termina haciendo un \code{toString()}.

El que es bastante inusual es el cuarto caso. Luego de haber visto el segundo, uno tiende a esperar que el cuarto de el mismo resultado, pero esto no es así. ?`Qué pasó? El intérprete consideró a la primera parte de la expresión como un bloque de código vacío y la ignoró. Luego, la expresión quedó como \code{+[]}, donde el \code{+} representa el operador unario, no el binario. Lo que significa que resultó en una coerción hacia \code{number} del string vacío \code{''}, resultando en un \code{0}. De hecho, podemos corroborarlo haciendo uso de los paréntesis, entonces el intérprete lo leerá como una expresión y \code{\{\}} representará un objeto en vez de un  bloque.

\begin{lstlisting}[title={Caso especial}]
({} + [])			// '[object Object]'
\end{lstlisting}

Con todo esto visto, ahora se tiene un mejor panorama de cómo funciona el operador \code{+} en Javascript. Se utiliza principalmente en valores de tipos numéricos o strings. 

Para el caso de los objetos, si alguno de los operandos es string, el mismo se convierte a string, en caso contrario, se convierte a número. 

Para el caso de los arreglos, el operador \code{+} no funciona como en otros lenguajes. Para concatenar arreglos es necesario utilizar el método \code{concat} o hacer uso de alguna librería externa.

\section{Operadores de igualdad}
\label{sec:eqeqeq}

Una de las primeras fallas comunes en cuanto a los programadores que se acercan a Javascript y vienen de otros lenguajes de la rama de C, es pensar que el operador de comparación \code{==} funciona de la misma manera que en esos otros lenguajes. Esto no es así. En Javascript, existen dos operadores de comparación de igualdad, estos son \code{==} y \code{===} (junto con sus comparadores de desigualdad análogos, \code{!=} y \code{!==}).

El operador de igualdad \code{===} se le suele llamar estricto, ya que si los dos operandos son de distintos tipos, la expresión será \code{false}, en caso contrario, hará una comparación del valor de los operandos (a excepción de los objetos, que compara referencias).

En cambio el operador de igualdad \code{==} se le suele llamar blando, o según la especificación, comparador de igualdad abstracta. Si los operandos son de distintos tipos, intentará convertir los tipos para luego compararlos.

\textbf{COMPLETAR}

La falta de "`transitividad"' en el operador \code{==} es una de las cosas más preocupantes del lenguaje. Es fácil pensar que si \code{A == B} y que \code{B == C}, entonces \code{A == C}, pero esto no sucede siempre.  Por ejemplo:

\begin{lstlisting}[title={Falta de transitividad en \code{==}}]
0 == ''             // true
0 == '0'            // true
'' == '0'           // false

false == undefined  // false
false == null       // false
null == undefined   // true
\end{lstlisting}

El consejo de los autores experimentados, es siempre usar el operador \code{===}.

