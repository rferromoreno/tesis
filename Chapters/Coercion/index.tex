\chapter{Coercion} % Main chapter title

\label{ch:coercion} % For referencing the chapter elsewhere, use \ref{Chapter1} 

En este capítulo se hablará de los efectos de la conversión de tipos que ocurre en el lenguaje. Tanto la implícita como la explícita. Cuando un lenguaje tiene seguridad en su sistema de tipos, hace que el mismo sea coherente, predecible, que sea fácil de intuir los tipos que se manejan. Para que un lenguaje gane flexibilidad en su sistema de tipos, debe sacrificar éstos aspectos de seguridad. Sin embargo, se pierden también los calificativos recién mencioandos.

Una de las mayores críticas al lenguaje es sobre la conversión de tipos implícita que ocurre en las expresiones. Tiene la fama de ser un lenguaje extremadamente flexible, pero al mismo tiempo impredecible en cuanto a la coerción.

