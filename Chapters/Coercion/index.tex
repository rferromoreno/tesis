\chapter{Coercion} % Main chapter title

\label{ch:coercion} % For referencing the chapter elsewhere, use \ref{Chapter1} 

Cuando un lenguaje tiene seguridad en su sistema de tipos, hace que el mismo sea coherente, predecible, que sea fácil de intuir los tipos que se manejan. Para que un lenguaje gane flexibilidad en su sistema de tipos, debe sacrificar éstos aspectos de seguridad. Sin embargo, se pierden también los calificativos recién mencionados.

Una de las mayores críticas a Javascript es sobre la conversión de tipos implícita que ocurre en las expresiones. Tiene la fama de ser un lenguaje extremadamente flexible, pero al mismo tiempo impredecible en cuanto a la coerción. En este capítulo se hablará de los efectos de la conversión de tipos que ocurre en el lenguaje, tanto la implícita como la explícita, y así podremos acercarnos a una conclusión de qué tan relajado es el sistema de tipos con respecto a la coerción.

\section{Operadores de igualdad}
\label{sec:eqeqeq}

Una de las primeras fallas comunes en cuanto a los programadores que se acercan a Javascript y vienen de otros lenguajes, es pensar que el operador de comparación \code{==} funciona de la misma manera que en otros lenguajes. Esto no es así. En Javascript, existen dos operadores de comparación de igualdad, estos son \code{==} y \code{===} (junto con sus comparadores de desigualdad análogos, \code{!=} y \code{!==}).



La falta de "`transitividad"' en el operador \code{==} es una de las cosas más preocupantes del lenguaje. Es fácil pensar que si \code{A == B} y que \code{B == C}, entonces \code{A == C}, pero esto no sucede siempre.
 

