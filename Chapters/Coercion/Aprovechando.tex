\section{Aprovechando la coerción}
\label{sec:aprovechando}

Hasta este punto se ha hablado de los efectos inesperados o con poca coherencia que tiene la coerción en el lenguaje. Sin embargo, cuando uno conoce las reglas que tiene el lenguaje, puede empezar a hacer uso de esos efectos a su favor.

Recordemos principalmente que Javascript es, entre otras cosas, un lenguaje de "`scripting"', por lo que se espera que haya ciertas facilidades y no tanta burocracia delante del programador a la hora de realizar su tarea.

Si bien la coerción lleva a efectos inesperados, bugs y dificultad a la hora de trazar y encontrar errores en el código, es probable que los programadores experimentados destaquen el punto de que se pueden hacer cosas muy concisas.

Por ejemplo, supongamos un escenario de una llamada asincrónica donde nos puede venir \code{data}, asi como también puede faltar dicha información (que no nos llegue dicha propiedad), luego en el código podemos verificar que la misma "`existe"' mediante el uso de la coerción en una expresión.

\begin{lstlisting}
if (data) {
	// hacer operaciones
}
\end{lstlisting}

Otro ejemplo puede ser el de verificar si una función está definida o no en cuanto a su valor de verdad. Supongamos que tenemos tenemos una función que tiene un parámetro opcional \code{callback} y si ésta fue definida, queremos ejecutarla.

\begin{lstlisting}
function procesar(callback) {
	// hacer operaciones
	callback && callback();
}
\end{lstlisting}

La línea 3 representaría un chequeo de "`si callback fue definido, entonces ejecutarla"'. Si \code{callback} es \code{undefined} (porque se llamó a \code{procesar} sin argumentos), entonces la parte izquierda de esa expresión dará \code{false} y el operador \code{\&\&} hará un "`cortocircuito"', evitando ejecutar la segunda parte.

Estos dos ejemplos sirven para presentar al lector que, dentro del lenguaje, la coerción puede ser utilizada a favor del programador.