\section{Conclusiones}

\begin{itemize}
\item Lo bueno: 
JavaScript no será un lenguaje de programación funcional pura, pero el hecho de que las funciones sean valores dentro del lenguaje lo hace excesivamente poderoso. Poseer funciones como valores, permitir el paso de las mismas como argumentos o como valores de retorno hace al lenguaje sumamente expresivo.
Para la enseñanza de conceptos básicos de la programación funcional, con una sintaxis similar a la familia de lenguajes de C, es un buen lenguaje, más allá de la falta de todas las propiedades que corresponden a un lenguaje 100\% del paradigma funcional.
\item Lo malo: 
La falta de evaluación perezosa, semántica de valores, transparencia referencial, entre otras cosas, lo alejan del paradigma funcional. Sin embargo, si tuviera éstas características seguramente tendríamos que limitar al lenguaje en otros aspectos, y probablemente quitarle el soporte para otros paradigmas de programación. 
\end{itemize}