\section{El operador \code{typeof}}

Para el análisis del tipo de un valor (o del valor ligado a una variable) se puede hacer uso del operador \code{typeof}. Dicho operador retorna un \code{string} con el nombre del tipo del valor evaluado.

\begin{lstlisting}[title={Analizando los tipos con \code{typeof}}]
typeof undefined		// "undefined"
typeof 123					// "number"
typeof true					// "boolean"
typeof {}						// "object"
typeof "Hola Mundo" // "string"
typeof Symbol()			// "symbol"
\end{lstlisting}

Una de las primeras flaquezas presentadas por el lenguaje es la del \code{typeof null}. 

\begin{lstlisting}[title={Analizando \code{typeof null}}]
typeof null					// "object"
\end{lstlisting}

Se tiende a esperar que \code{typeof null} retorne \code{``null''}, sin embargo, retorna \code{``object''}. Este es un \textit{bug} conocido y difícilmente sea corregido, ya que se estima que hay muchas aplicaciones y sistemas en la web que se basan en este comportamiento. Se cree que corregir esto crearía más problemas que soluciones.

Esto no solo pasa con el literal \code{null} sino que además sucederá con cualquier variable ligada a un tipo nulo.

\begin{lstlisting}[title={Analizando \code{typeof null} (cont.)}]
var a = null;
typeof a					// "object"
\end{lstlisting}

\subsection{Casos especiales del \code{typeof}}

Existen algunos casos especiales para el operador \code{typeof}. ?`Qué pasa con las funciones?. ?`Y con los arreglos?. 

El primero de los casos es el de las \textbf{funciones}. Como se mencionó anteriormente, en la especificación, una función es considerada un subtipo de \code{object}, a diferencia de que tiene una propiedad interna \code{[[Call]]}. Sin ir más lejos, ?`qué se espera que retorne \code{typeof} para el caso de una función?

\begin{lstlisting}[title={Analizando \code{typeof} de una función}]
typeof function a(){}					// "function"
\end{lstlisting}

Si bien quizás resulte más intuitivo esperar que retorne \code{``object''}, el hecho de que haya retornado \code{``function''} puede resultar útil a la hora de distinguir entre objetos y funciones, y así identificar cuales son los que se pueden invocar (también conocidos como "`callable objects"'). Sin embargo, este hecho es algo contradictorio ya que \code{``function''} no está distinguido entre los tipos primitivos.

El otro caso especial es el de los \textbf{arreglos}. En JavaScript, un arreglo no es más que un objeto con una propiedad interna \code{length}, donde cada propiedad del objeto es el índice del arreglo.

\begin{lstlisting}[title={Analizando \code{typeof} de arreglos}]
typeof []									// "object"
typeof [1, 2, 3]					// "object"
typeof ["hola", "mundo"]	// "object"
\end{lstlisting}

?`Por qué para las funciones el operador \code{typeof} devuelve \code{``function''} mientras que para los arreglos sigue devolviendo \code{``object''}? 

Al parecer, la distinción de los "`callable objects"' es importante para el lenguaje, pero para el caso de arreglos es irrelevante saber si un arreglo es efectivamente un arreglo o simplemente un objeto, dado que tienen las mismas propiedades y se lo puede tratar de la misma manera. ?`Cómo saber entonces cuando -por ejemplo- una variable es un objeto o un arreglo? La respuesta es mediante el operador \code{instanceof}. La distinción se hace mediante el análisis de la clase asociada, y no del tipo.

Para cerrar, existe un caso curioso para el valor especial numérico \code{NaN}. Recordemos que este es un valor especial para representar "`Not a Number"' (del español, \textit{No es un número}). Sin embargo cuando hacemos \code{typeof} del mismo, obtenemos que el mismo es de tipo numérico.

\begin{lstlisting}[title={Analizando \code{typeof} de \code{NaN}}]
typeof NaN								// "number"
\end{lstlisting}

Esto sucede lógicamente porque, en el fondo, \code{NaN} es de hecho un \code{number}. Tal como se mencionó anteriormente, JavaScript implementa la norma de punto flotante (IEEE 754), la cual tiene una \textit{excepción} para determinados valores que en la aritmética de punto flotante son inválidos.