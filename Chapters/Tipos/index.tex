\chapter{Tipos} % Main chapter title

\label{ch:tipos} % For referencing the chapter elsewhere, use \ref{Chapter1} 

%----------------------------------------------------------------------------------------

JavaScript es un lenguaje de "`scripting"', con tipado dinámico y un sistema de tipos débil. Generalmente, en este tipo de lenguajes interpretados no es necesario definir el tipo de una variable al momento de declararla. Sin embargo, a veces es necesario cuestionar a los tipos que posee un lenguaje para conocerlos mejor. En este capítulo se abordará las cuestiones de JS relativas a los tipos del lenguaje. También se hará un análisis especial del operador \code{typeof}, en donde podremos comenzar a cuestionar flaquezas del lenguaje.

\section{Tipos primitivos}

Tal como se mencionó en el Capítulo \ref{ch:introduccionjs}, existen solamente 7 tipos primitivos del lenguaje. Una variable está asociada a un valor, y dicho valor puede ser de alguno de los siguientes tipos:

\begin{itemize}
\item \code{undefined}
\item \code{null}
\item \code{number}
\item \code{string}
\item \code{boolean}
\item \code{symbol}
\item \code{object}
\end{itemize}

Algunos autores consideran que \code{Object} no es un tipo primitivo. De hecho, en la sección \ref{sec:toprimitive} (y a lo largo del capítulo \ref{ch:coercion}) mostraremos que la especificación hace una distinción especial para éste tipo a la hora de hacer conversiones de tipo, por lo que pensar que \code{Object} no es un tipo primitivo da lugar a debate. 

También existe un concepto errado de que "`en JavaScript todo es un objeto"'. Considerando que en JavaScript las funciones y los arreglos son objetos, la frase es en parte cierta. Pero el concepto es equivocado porque además de objetos existen los otros tipos primitivos, de los cuales haremos mención en esta sección.

En las siguientes subsecciones haremos introducción (y algunas críticas) a algunos de los tipos. No son tenidos en cuenta \code{boolean}, dado que no hay mucho que analizar, y tampoco \code{symbol}, el cual no formará parte de éste documento.

\subsection{\code{null} y \code{undefined}}

Estos dos tipos representan el valor "`vacío"', y puede resultar confuso, porque tener dos identificadores para \textit{casi} lo mismo parece, desde el primer momento, redundante. El valor de \code{undefined} se suele utilizar como valor para variables que aún no fueron declaradas, mientras que el valor de \code{null} se supone que cuando queremos forzar a que una variable tenga valor nulo.

\begin{lstlisting}[title={\code{null} y \code{undefined}}]
var a;
var b = null;

console.log(a);	// undefined
console.log(b);	// null
\end{lstlisting}

El problema de \code{undefined} es que no existe un mecanismo para restringir el uso de este tipo. De hecho, podemos asignarle \code{undefined} como valor a una variable, y en ese sentido no podemos saber si fue porque se le asignó el valor o porque nunca obtuvo uno.

\begin{lstlisting}
var a;
var b = null;
b = undefined;

console.log(a);	// undefined
console.log(b);	// undefined
\end{lstlisting}

Aquí es donde entra el juego entre un valor "`no definido"' y un valor "`no declarado"'. Lo lógico, es pensar que \code{undefined} se utiliza para una variable que sí fue declarada, pero que su valor aún no fue definido. Mientras que una variable "`no declarada"' representaría a una variable que en ningún momento se la declaró mediante ningún mecanismo (ni \code{var}, ni \code{let}, ni \code{const}).

El problema es que el intérprete tampoco es muy concreto a la hora de describir los errores, ya que si la variable no fue declarada, !`nos dirá que la misma no fue definida!

\begin{lstlisting}
var a;

a; // undefined
b; // ReferenceError: b is not defined
\end{lstlisting}

Por el lado de \code{null} no hay mucho que aclarar. En el capítulo \ref{ch:coercion} observaremos cómo \code{null} es un valor de falsedad, pero al compararlo con \code{false}, dicha comparación da un resultado negativo.

\subsection{\code{string}}

El tipo \code{string} es el utilizado para representar una cadena de caracteres o texto. En otros lenguajes, generalmente un \code{string} es tratado como una lista, una secuencia o un array de caracteres (porque de hecho, en algunos lenguajes lo es). Sin embargo, en JavaScript no existe el concepto de \code{char}. Solamente el de \code{string}. Si se desea, en JavaScript se puede tratar a un \code{string} como si fuera un \code{array}, pero no lo es. Solamente poseen varios "`métodos en común"'.

\begin{lstlisting}[title={Similitudes entre \code{string} y un \code{array}}]
var a = 'hola';                 // string
var b = ['h', 'o', 'l', 'a'];   // array

console.log(a.length);          // 4
console.log(b.length);          // 4

console.log(a.indexOf('o'))     // 1
console.log(b.indexOf('o'))     // 1

console.log(a.concat('mundo'))  
// holamundo
console.log(b.concat(['m','u','n','d','o']))
// ​​​​​[ 'h', 'o', 'l', 'a', 'm', 'u', 'n', 'd', 'o' ]​​​​​
\end{lstlisting}

A pesar de que existe cierta similitud entre estos dos tipos, es un error pensar a \code{string} como una simple cadena de caracteres. Observemos:

\begin{lstlisting}
var a = 'hola';                 // string
var b = ['h', 'o', 'l', 'a'];   // array

a[0] = 'H';
b[0] = 'H';

console.log(a);		// hola
console.log(b);		// ​​​​​[ 'H', 'o', 'l', 'a' ]​​​​​
\end{lstlisting}

El valor de la primera letra de la variable \code{a} no se vio modificada. Esto sucede porque en JavaScript el tipo \code{string} es inmutable. Las operaciones que vienen con la clase \code{String} devuelven un nuevo valor, sin mutar ni cambiar el valor con el cual operamos.

Por otro lado, existen algunos métodos que existen para la clase \code{Array} pero no para \code{String}. Serían deseables métodos como \code{join}, \code{map} o \code{reverse}. Por suerte existen formas de tomar esos métodos "`prestados"': haciendo conversión hacia \code{array} y volviendo a \code{string} (poco óptimo) o forzando las llamadas mediante el uso del método \code{call}, del cual se hablará con precisión en la sección \ref{subsec:ligaduraexplicita}.

Lo cierto, es que el tipo \code{string} no es un arreglo de caracteres, y en cierto punto el programador tendrá que evaluar si resultará una mejor solución a su problema utilizar este tipo, o un \code{array} de \code{string}, en caso de que tenga datos que deba ir "`mutando"'.

\subsection{\code{number}}

El otro tipo a destacar en el lenguaje es \code{number}. En JavaScript no existe una distinción entre \code{integer} y \code{float}, \code{real} o \code{double}. Solamente existe el concepto de \code{number} y éste es compartido, independientemente si se trata de un número entero o real.

Existe una famosa crítica al lenguaje que se suele ver en Internet, que es la siguiente:

\begin{lstlisting}
0.1 + 0.2 === 0.3			// false
\end{lstlisting}

Sin embargo, criticar a JavaScript por dicha expresión es una equivocación enorme. De hecho, este problema también lo poseen lenguajes como C, C++, Java, Python, PHP, entre otros (ver \href{https://0.30000000000000004.com/}{0.30000000000000004.com}). 

En todo caso, se puede criticar al sistema de tipos (por utilizar únicamente \code{number} y no tener un manejo natural para los decimales "`grandes"'). Pero lo que realmente sucede bajo esa expresión, es que el tipo \code{number} en JavaScript implementa la norma IEEE 754 de punto flotante, por lo que algunas veces habrá cierta pérdida de precisión en algunas operaciones matemáticas para determinados valores. 

Por otro motivo que se suele cuestionar a JavaScript es por los valores numéricos especiales: \code{+0}, \code{-0}, \code{Infinity}, \code{-Infinity} y \code{NaN}. Pero de nuevo, estamos en lo mismo: Todos estos valores están ahí porque forman parte de la implementación de IEEE 754.

De lo que sí hay que hacer mención es de la parte sintáctica de los números. Existen diversas formas de escribir números (y algunos de ellos representan lo mismo). A continuación se presentan algunos ejemplos. Todos de ellos son formas válidas.

\begin{lstlisting}
// ejemplo 1
42
42.0
42.

// ejemplo 2
42.3
42.300

// ejemplo 3
0.5
.5
\end{lstlisting}

La parte "`desagradable"' para algunos puede ser la de la línea 4 y la línea 12, donde se usa el punto al principio o al final del literal. 

Un punto fuerte es que se pueden usar literales en notación científica o en otras bases, como hexadecimal, octal o binario. Éste último concepto fue introducido en versiones previas pero mejorado en ES6.

\begin{lstlisting}
// Notación científica
1E3			// 1 * 10^3
1.1E6		// 1.1 * 10^6
2e-5		// 2 * 10^-5

// Hexadecimal
0xf3; // 243
0Xf3; // 243

// Octal
0o363;		// 243
0O363;		// 243

// Binario
0b11110011;	// 243
0B11110011; // 243
\end{lstlisting}

\subsection{\code{object}}

\textbf{COMPLETAR}


\section{El operador \code{typeof}}

Para el análisis del tipo de un valor (o del valor de una variable) en ejecución, se puede hacer uso del operador \code{typeof}. Dicho operador retorna un String con el nombre del tipo del valor evaluado.

\begin{lstlisting}[title={Analizando los tipos con \code{typeof}}]
typeof undefined		// "undefined"
typeof 123					// "number"
typeof true					// "boolean"
typeof {}						// "object"
typeof "Hola Mundo" // "string"
typeof Symbol()			// "symbol"
\end{lstlisting}

Una de las primeras flaquezas presentadas por el lenguaje es la del \code{typeof null}. 

\begin{lstlisting}[title={Analizando \code{typeof null}}]
typeof null					// "object"
\end{lstlisting}

Uno tiende a esperar que \code{typeof null} retorne "\code{null}", sin embargo, retorna "\code{object}". Este es un \textit{bug} conocido y difícilmente sea corregido, ya que se estima que hay muchas aplicaciones y sistemas en la web que se basan en este comportamiento. Se cree que corregir esto crearía más problemas que soluciones.

Esto no solo pasa con el literal \code{null} sino que además sucederá con cualquier variable ligada a un tipo nulo.

\begin{lstlisting}[title={Analizando \code{typeof null} (cont.)}]
var a = null;
typeof a					// "object"
\end{lstlisting}

\subsection{Casos especiales del \code{typeof}}

Existen algunos casos especiales para el operador \code{typeof}. ?`Qué pasa con las funciones?. ?`Y con los arreglos?. ?`Y con los tipos built-in pero que no son primitivos (también conocido como "`nativos"')?. Vamos por partes.

El primero de los casos es el de las \textbf{funciones}. Como se mencionó anteriormente, en la especificación, una función es considerada un subtipo de \code{object}, a diferencia de que tiene una propiedad interna \code{[[Call]]}. Sin ir más lejos, ?`qué se espera que retorne \code{typeof} para el caso de una función?

\begin{lstlisting}[title={Analizando \code{typeof} de una función}]
typeof function a(){}					// "function"
\end{lstlisting}

Si bien quizás resulte más intuitivo esperar que retorne "\code{object}", el hecho de que haya retornado "\code{function}" puede resultar útil a la hora de distinguir entre objetos y funciones, y así identificar cuales son los que se pueden invocar (también conocidos como "`callable objects"'). Sin embargo, este hecho es algo contradictorio ya que "\code{function}" no está distinguido entre los tipos primitivos.

El otro caso especial es el de los \textbf{arreglos}. En JavaScript, un arreglo no es más que un objeto con una propiedad interna \code{length}, donde cada propiedad de la instancia del objeto es el índice del arreglo.

\begin{lstlisting}[title={Analizando \code{typeof} de arreglos}]
typeof []									// "object"
typeof [1, 2, 3]					// "object"
typeof ["hola", "mundo"]	// "object"
\end{lstlisting}

?`Por qué para las funciones el operador \code{typeof} devuelve \code{function} mientras que para los arreglos sigue devolviendo \code{object}? 

Al parecer, la distinción de los "`callable objects"' es importante para el lenguaje, pero para el caso de arreglos es irrelevante saber si un arreglo es efectivamente un arreglo o simplemente un objeto, dado que tienen las mismas propiedades y se lo puede tratar de la misma manera. ?`Cómo saber entonces cuando -por ejemplo- una variable es un objeto o un arreglo? La respuesta es mediante el operador \code{instanceof}. La distinción se hace mediante el análisis de la clase asociada, y no del tipo.

Para cerrar, existe un caso curioso para el valor especial numérico \code{NaN}. Recordemos que este es un valor especial para representar "`Not a Number"' (del español, \textit{No es un número}). Sin embargo cuando hacemos \code{typeof} del mismo, obtenemos que el mismo es de tipo numérico.

\begin{lstlisting}[title={Analizando \code{typeof} de \code{NaN}}]
typeof NaN								// "number"
\end{lstlisting}

Esto sucede lógicamente porque, en el fondo, \code{NaN} es de hecho un \code{number}. Tal como se mencionó anteriormente, JavaScript implementa la norma de punto flotante (IEEE 754), la cual tiene una \textit{excepción} para valores que para la aritmética de punto flotante son inválidos.