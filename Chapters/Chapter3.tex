\chapter{Paradigma funcional}

\label{Chapter3}

% -----

Entre los puntos fuertes que posee el lenguaje, es indiscutible decir que las funciones son uno de los ejes principales. Sin embargo, ¿qué tan ligado está el lenguaje al paradigma de programación funcional?. ¿Soporta todas sus características?.

Es sabido que algunos conceptos como las funciones de alto orden están vinculadas con el paradigma funcional, mientras que también se conoce que JavaScript permite el pasaje de funciones como parámetros, o el retorno de las mismas como valores. En este capítulo se realizará un análisis sobre las mismas, para poder entender qué tan cerca o lejos está el lenguaje de las características del paradigma.

% -----

\section{Funciones puras e híbridas}

\section{Efectos colaterales}

\section{Recursividad}

\section{Funciones de primera clase y orden superior}

\section{Evaluación estricta y perezosa}