\chapter{Paradigma funcional}

\label{Chapter3}

% -----

Entre los puntos fuertes que posee el lenguaje, es indiscutible decir que las funciones son uno de los ejes principales. Sin embargo, ¿qué tan ligado está el lenguaje al paradigma de programación funcional?. ¿Soporta todas sus características?.

Es sabido que algunos conceptos como las funciones de alto orden están vinculadas con el paradigma funcional, mientras que también se conoce que JavaScript permite el pasaje de funciones como parámetros, o el retorno de las mismas como valores. En este capítulo se realizará un análisis sobre las mismas, para poder entender qué tan cerca o lejos está el lenguaje de las características del paradigma.

% -----

\section{Recursividad}

Al igual que los lenguajes sintáticamente similares a JavaScript, la recursión es una técnica que se aplica con naturaleza. Siempre que la función no sea anónima (es decir, que tenga un nombre), se puede aplicar recursión directa sin problemas. Además, el soporte para la recursión mutua también es posible gracias a la característica de \textit{hoisting}.

\section{Funciones puras}

Las funciones puras son otra de las claves del paradigma funcional. Esto es, funciones que bajo la misma entrada, siempre devuelven el mismo resultado.

Lamentablemente el lenguaje no posee ninguna directiva, y tampoco existe una herramienta concreta para determinar la pureza de la función, más que el conocimiento del programador.

A continuación, se mencionan algunas prácticas para evitar las funciones impuras

\begin{itemize}
  \item Evitar el uso de variables que estén fuera del ámbito (scope) de la función.
  \item Evitar el uso del DOM o de variables del browser como \code{document} o \code{window}
\end{itemize}


\section{Funciones de primera clase y orden superior}

Una de las características más destacables del lenguaje es que las funciones son objetos, que a su vez se pueden almacenar como valores. Gracias a esto, resulta extremadamente simple el pasaje de funciones como argumentos, así como también devolver funciones como un valor de retorno.

\section{Evaluación estricta y perezosa}