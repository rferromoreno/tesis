\chapter{Paradigma funcional}

\label{Chapter3}

% -----

Entre los puntos fuertes que posee el lenguaje, es indiscutible decir que las funciones son uno de los ejes principales. Sin embargo, ?`qué tan ligado está el lenguaje al paradigma de programación funcional?. ?`Soporta todas sus características?.

Es sabido que algunos conceptos como las funciones de alto orden están vinculadas con el paradigma funcional, mientras que también se conoce que JavaScript permite el pasaje de funciones como parámetros, o el retorno de las mismas como valores. En este capítulo se realizará un análisis sobre las mismas, para poder entender qué tan cerca o lejos está el lenguaje de las características del paradigma.

% -----

\section{Recursividad}

Al igual que los lenguajes sintáticamente similares a JavaScript, la recursión es una técnica que se aplica con naturaleza. Siempre que la función no sea anónima (es decir, que tenga un nombre), se puede aplicar recursión directa sin problemas. Además, el soporte para la recursión mutua también es posible gracias a la característica de \textit{hoisting}.

Cualquiera de las formas vistas en el Capítulo 1 son válidas para definir una función recursiva. A continuación se presentan ejemplos de las mismas, mostrando recursión directa y cruzada:

\begin{lstlisting}[title={Ejemplos de funciones recursivas}]
// Función recursiva con función como declaración
function sumatoria(n) {
  return n > 0 ? n + sumatoria(n - 1) : 0;
}

// Función recursiva con función como expresión
var factorial = function(n) {
  return n > 0 ? n * factorial(n - 1) : 1;
};

// Recursión cruzada
var esPar = num => (num === 0 ? true : esImpar(num - 1));
var esImpar = num => (num === 0 ? false : esPar(num - 1));
\end{lstlisting}

\section{Funciones puras}

Las funciones puras son otra de las claves del paradigma funcional. Esto es, funciones que bajo la misma entrada, siempre devuelven el mismo resultado. Esta cualidad brinda la propiedad de transparencia referencial entre los rasgos de un lenguaje perteneciente al paradigma funcional. Los beneficios que trae el uso de las funciones puras, son la testeabilidad, predictibilidad, y la falta de efectos colaterales.

Por suerte, en el lenguaje se pueden escribir fácilmente funciones puras. Lamentablemente el lenguaje no posee ninguna directiva, y tampoco existe una herramienta concreta para determinar la pureza (o impureza) de una función, más que el conocimiento del programador.

Algunos ejemplos de funciones puras se pueden ver a continuación:

\begin{lstlisting}[title={Funciones puras}]
function sumar(a, b) {
  return a + b;
}

function esMayor(edad) {
  return edad >= 18;
}

function calcularPrecio(cantidad, costo) {
	return cantidad * costo;
}
\end{lstlisting}

?`Por qué son funciones puras las presentadas previamente? Porque no causan efectos colaterales, y dependen únicamente de sus argumentos para generar un resultado. Ante los mismos argumentos, dichas funciones retornarán los mismos valores. Sin embargo, tal como mencionamos, es muy fácil caer en el uso de funciones impuras:

\begin{lstlisting}[title={Funciones impuras}]
function obtenerDiaActual() {
  return new Date().getDate();
}

function obtenerTextoPorId(id) {
  return document.getElementById(id).textContent;
}

const PI = Math.PI;

function calcularArea(radio) {
  return PI * radio * radio;
}
\end{lstlisting}

La función \code{obtenerDiaActual} retorna un valor "`azaroso"', y la misma llamada a la función puede retornar dos valores diferentes (dependiendo de la hora de la computadora). Para el segundo caso, \code{obtenerTextoPorId} depende necesariamente del estado de \code{document} que representa el DOM del navegador. Algo similar sucede con \code{calcularArea}, incluso habiendo declarado a \code{PI} como una "`constante"', la función (de manera aislada) depende de un valor que está por fuera de su scope local, y no nos podemos asegurar de la ausencia de un efecto colateral.

Si bien el lenguaje no provee ningun mecanismo para identificar o limitar únicamente las funciones puras, a continuación se mencionan algunas prácticas para evitar las funciones impuras:

\begin{itemize}
  \item Evitar el uso de variables que estén fuera del ámbito (scope) de la función. Esto incluye también a las variables globales del navegador.
  \item Evitar el uso del DOM o de variables del browser como \code{document} o \code{window}.
	\item Evitar el uso de \code{Math.random}, valores azarosos o cambiantes como el día y la hora actual.
	\item Evitar peticiones bajo el protocolo HTTP u otras que dependan del estado de la red.
	\item Evitar imprimir por pantalla o por consola, o cualquier mecanismo de I/O.
	\item Evitar la mutabilidad de datos. Recordar que los objetos en JS se manejan por referencia. Para éste ítem, se recomienda aplicar prácticas de inmutabilidad, ya sea mediante \code{Object.freeze} o el uso de alguna librería externa.
\end{itemize}

\section{Funciones de primera clase y orden superior}

Una de las características más destacables del lenguaje es que las funciones son objetos, que a su vez se pueden almacenar como valores. Gracias a esto, resulta extremadamente simple el pasaje de funciones como argumentos, así como también devolver funciones como un valor de retorno.

\subsection{Funciones de primera clase}

En JavaScript las funciones son objetos de primera clase, dado que las funciones son tratadas como cualquier otro valor. Al momento de crear una función, en realidad lo que se crea es un objeto de clase \code{Function}.

Para el caso de una función mediante declaración, se asignará a una variable ligada al nombre dado a la función (siempre que no sea anónima). Para el caso de la función por expresión, dependerá si la misma se realiza en una asignación.

\begin{lstlisting}[title={Analizando el valor de una función}]
function saludar() { console.log('hola'); }
var despedirse = () => console.log('chau');

console.log(saludar);     // [Function: saludar]
console.log(despedirse);  // [Function: despedirse]

console.log(saludar.toString());  
// function saludar() { console.log('hola'); }
console.log(despedirse.toString()); 
// () => console.log('chau')
\end{lstlisting}

El hecho de que las funciones sean tratadas como valores, a su vez, hace que se puedan crear objetos que dentro de sus pares clave-valor tengan propiedades que sean funciones. Es decir, métodos del objeto.

\begin{lstlisting}[title={Asignando una función como valor de una propiedad a un objeto}]
var gato = {
  nombre: 'Mishi',
  maullar: function() {
    console.log('miau');
  }
};

gato.maullar();   
// miau
console.log(gato);  
// { nombre: 'Mishi', maullar: [Function: maullar] }
\end{lstlisting}

\subsection{Funciones de orden superior}

Dado que las funciones son tratadas como cualquier otro valor, el lenguaje permite la característica de que una función pueda ser pasada como argumento, o que sea el valor de retorno de otra función. Esto conlleva a alcanzar otros mecanismos, como es el curry en las funciones o la aplicación parcial.

Comencemos con algunos ejemplos:

\begin{lstlisting}[title={Pasando una función como argumento}]
var saludar = () => console.log('hola mundo!');

function ejecutar(fn) {
  fn();
}

function ejecutarSeguro(fn) {
  if (typeof fn === 'function') {
    fn();
  }
}

ejecutar(saludar);
ejecutarSeguro(saludar);
\end{lstlisting}

En el código mostrado, se poseen dos funciones, \code{ejecutar} y \code{ejecutarSeguro}. La única diferencia es que éste último hace un chequeo de tipo del argumento brindado para asegurarse de que es una función y puede invocarla. Para el primer caso, si no le dieramos una función como parámetro, la ejecución de dicha llamada resultaría en un error.

La otra cualidad es la de retornar funciones, la cual se ejemplificará a continuación:

\begin{lstlisting}[title={Retornando funciones}]
function generador(prefijo) {
  return texto => console.log(prefijo + ' ' + texto);
}

var imprimir = generador('Hola');

// En este momento, imprimir es una función!
console.log(typeof imprimir); // function

imprimir('mundo');    // Hola mundo
imprimir('che!');     // Hola che!
imprimir('a todos');  // Hola a todos
\end{lstlisting}

La función \code{generador} tiene un parámetro \code{prefijo} y retorna una función, que tiene un parámetro \code{texto}, y muestra mediante la consola la concatenación del prefijo y del texto. Dado que \code{generador} retorna una función, el valor de la misma es ligada a \code{imprimir}, por lo que tiene sentido pensar que éste es como un alias de la función y se puede invocar sin ningún inconveniente. En el ejemplo, se puede observar como \code{imprimir} de alguna forma "`recuerda"' el prefijo que fue establecido al momento de "`generar la función"'.

Veamos otro ejemplo, haciendo una función suma que esté en una forma \textit{curried}:

\begin{lstlisting}[title={Función suma \textit{curried}}]
var sumar = a => b => a + b;

var sumar2 = sumar(2);
var sumar5 = sumar(5);

console.log(sumar2(10));	// 12
console.log(sumar5(10));	// 15
\end{lstlisting}

Si bien la sintaxis de las \textit{arrow functions} al principio puede resultar ajena, la función sumar es una función con un argumento \code{a} y retorna una función, que tiene un argumento \code{b}, la cual retorna la suma de \code{a + b}. De alguna forma, se puede observar como \code{sumar2} y \code{sumar5} son una aplicación parcial de la función \code{sumar}.

Otra forma de realizar aplicación parcián es mediante el método \code{bind} de la clase \code{Function}, el cual recibe como primer argumento el contexto (por lo general sería \code{this}), y luego una lista de argumentos, para retornar la función aplicada (total o parcialmente). Supongamos el siguiente código:

\begin{lstlisting}[title={Aplicación parcial usando \code{bind}}]
var multiplicar = (a, b) => a * b;

var duplicar = multiplicar.bind(null, 2);

console.log(duplicar(15));	// 30
\end{lstlisting}

En este caso, \code{multiplicar.bind} nos retornará una función que será producto de aplicar parcialmente un \code{2} como primer argumento. 

Las funciones de alto orden en el lenguaje son una de las características más poderosas que éste posee. La facilidad con la que se puede pasar o retornar una función brindan un mecanismo que otros lenguajes de programación no pueden alcanzar, y éste es uno de los puntos más fuertes de JavaScript.

\section{Evaluación ansiosa y perezosa}

Una característica deseable en los lenguajes del paradigma funcional es la de la evaluación perezosa (lazy). Sin embargo, la evaluación de expresiones en JavaScript se realiza de manera ansiosa (eager). Supongamos el siguiente código:

\begin{lstlisting}[title={Creando una función condicional}]
function elegir(condicion, opcion1, opcion2) {
  if (condicion) {
    console.log(opcion1);
  } else {
    console.log(opcion2);
  }
}
\end{lstlisting}

La función \code{elegir} es de alguna manera, la forma declarativa del \code{if-else}. Como primer parámetro recibe una condición, y como segundo y tercer parámetro recibe las opciones para el caso de que la condición sea verdadera o falsa.

Si el lenguaje implementara la evaluación perezosa, se espera que las expresiones no sean evaluadas hasta el momento de ser necesario (o postergar su ejecución al máximo). Es decir, si yo en \code{opcion1} u \code{opcion2} brindo como argumento una expresión, dicha expresión no debería ser evaluada si no entra por el flujo correspondiente del \code{if}. Si la condición es verdadera, \code{opcion2} no debe ser evaluada en ningún momento. Si la condición es falsa, entonces es \code{opcion1} la que no debe ser evaluada.

Por otro lado, aprovechando la libertad que tenemos en las expresiones gracias al sistema de tipos, podemos crear expresiones de la forma \code{true \&\& console.log()}, o también una expresión matemática como \code{2 + console.log()}. Cualquiera de estas formas son válidas, más allá de la coerción que ocurra en dichas expresiones, al momento de analizar la expresión, se imprimirá el mensaje dado por pantalla.

Dicho esto, ejecutamos las siguientes invocaciones:

\begin{lstlisting}[title={Analizando resultados de las invocaciones}]
elegir(true, 'a', 'b' && console.log('Ansioso!'));
// Ansioso!
// a

elegir(false, 2 + console.log('Ansioso!'), 5);
// Ansioso!
// 5
\end{lstlisting}

Como se puede analizar, para el primer caso, lo primero que ocurre es que se imprimime por pantalla el mensaje "`Ansioso!"', para luego imprimir "`a"'. En el segundo caso ocurre lo mismo: Primero se imprime "`Ansioso!"' y luego un "`5"'. Con esto, se puede mostrar que el lenguaje hace la evaluación de las expresiones al momento de hacer la invocación, y no al momento de que sea realmente necesario evaluar la expresión. En resumen, JavaScript posee evaluación ansiosa.

\section{Semántica de valores}

Por lo general, en los programas del paradigma funcional no se realizan asignaciones, y así, el valor asoaciado a una variable nunca cambia. Este concepto está directamente relacionado con el de funciones puras y transparencia referencial: Al no tener una dependencia sobre el valor de una variable, se espera que una función se evalúe siempre de la misma forma para una misma entrada.

Esta característica de los lenguajes del paradigma funcional da un enfoque puramente matemático, eliminando la noción de estado. Sin embargo esto no sucede en JavaScript. Una variable puede cambiar su valor. En el lenguaje existen sentencias, entre las cuales una de ellas es la asignación. Bajo esta premisa, entonces tendremos una noción de estado en las aplicaciones de nuestro lenguaje, por lo que podemos decir que JavaScript no cumple con ésta propiedad.

Dado el siguiente código:

\begin{lstlisting}[title={Noción de estado}]
function generarContador() {
  var contador = 0;
  return {
    contar: function() {
      contador++;
    },
    mostrarContador: function() {
      console.log(contador);
    }
  }
}

var obj = generarContador();

obj.contar();
obj.contar();
obj.mostrarContador();
\end{lstlisting}

Podemos observar un ejemplo de lo fácil que es poseer la noción del estado. Para el caso, el objeto tiene una variable interna \code{contador} el cual puede ir cambiando a medida que se vaya invocando el método \code{contar} sucesivas veces. En éste sentido, no se cumple con la semántica de valores.

\section{Conclusiones}

\begin{itemize}
\item Lo bueno: 
JavaScript no será un lenguaje de programación funcional pura, pero el hecho de que las funciones sean valores dentro del lenguaje lo hace excesivamente poderoso. 
\item Lo malo: 
La falta de evaluación perezosa, semántica de valores, transparencia referencial, entre otras cosas, son las características de ser un lenguaje puramente funcional. Sin embargo, si tuviera éstas características seguramente tendríamos que limitar al lenguaje en otros aspectos, y probablemente quitarle el soporte para otros paradigmas de programación.
\item Lo feo: 

\end{itemize}