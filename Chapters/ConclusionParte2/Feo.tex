\section{Lo feo}

Por el lado del paradigma funcional, si existiera un constructor natural para que los valores sean inmutables, servirían de extrema ayuda para acercarse aún más a dicho paradigma. Lamentablemente ésto no existe, entonces si queremos acercarnos a la programación funcional mediante JavaScript, tenemos que hacer uso de librerías externas.

Una de las características desagradables con respecto al paradigma orientado a objetos, es la falta de miembros privados, ya sean atributos o funciones, y tener que recurrir a aplicar mecanismos como closures e IIFEs para alcanzar esto. Todos los métodos y los atributos cargados en un objeto serán públicos y habrá que pensar en un buen diseño de aplicación para no tener problema con ello. Por suerte, los módulos de ES6 solucionan una parte de esto, ya que de forma implícita, los métodos que no se exporten en un módulo valdrán como métodos privados para ese módulo.