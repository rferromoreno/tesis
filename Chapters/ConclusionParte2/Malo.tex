\section{Lo malo}

Algunas carencias tales como la falta de evaluación perezosa, semántica de valores y transparencia referencial, alejan a JavaScript del paradigma funcional. A pesar de ello, si tuviera éstas características seguramente tendríamos que limitar al lenguaje en otros aspectos, y probablemente quitarle el soporte para otros paradigmas de programación. Si vamos al caso, agregar semántica de valores y transparencia referencial implica perder la noción de "`estado"', por lo que éste acercamiento al paradigma funcional nos obligaría a alejarnos del paradigma orientado a objetos.

Por el lado del paradigma de orientación a objetos, la falta de soporte natural para las clases es algo que deja que desear del lenguaje. Para ser justos con él, no fue pensado para tener clases (y es por eso que no tiene herencia clásica), y aún así los programadores buscan llegar a las mismas. Sin embargo, el soporte dado a las clases en ES6 es únicamente sintáctico, y el estándar parece estar conforme con dichas bases como para seguir evolucionando en éste punto. 

La asignación "`manual"' del prototipo de una función al querer simular la herencia clásica es un arma de doble filo, ya que si bien nos da libertades, es muy fácil perderse entre las relaciones que hay entre los objetos. Por otro lado, la poca popularidad de la herencia prototipal hace de JavaScript un lenguaje más difícil de comprender.