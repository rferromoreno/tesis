\section*{Lo bueno}

Por parte del paradigma de programación funcional, JavaScript no será un lenguaje de programación funcional pura, pero el hecho de que las funciones sean valores dentro del lenguaje lo hace excesivamente poderoso. Poseer funciones como valores, permitir el paso de las mismas como argumentos o como valores de retorno hace al lenguaje sumamente expresivo.

Para la enseñanza de conceptos básicos de la programación funcional, con una sintaxis similar a la familia de lenguajes de C, es un buen lenguaje, más allá de la falta de todas las propiedades que corresponden a un lenguaje 100\% del paradigma funcional.

Por el lado del paradigma de orientación a objetos, la herencia prototipal es una de las características destacadas. El modelo de delegación de comportamiento y la cadena de prototipo, y la falta de necesidad de copiar o guardar lugar en memoria para los miembros de la superclase seguramente lo hacen más liviano, pero a su vez con un extremo poder.
 
La sintaxis de clase de ES6 fue probablemente una de las mejores características lanzadas para el lenguaje. Permite a muchos programadores meterse en el lenguaje sin necesidad de cambiarles la mentalidad de herencia clásica a la prototipal. La incorporación de la sintaxis de \code{class} y \code{extends} fue sin dudas, un acierto.

En el sentido de la programación "`orientada a \textit{objetos}"', JavaScript es un verdadero lenguaje orientado a objetos, mientras que otros lenguajes son orientados a \textit{clases}. Volvemos a insistir sobre un algo marcado anteriormente: \textit{Casi todo en JavaScript es un objeto}: Funciones, arreglos, clases y módulos.

El polimorfismo (o en realidad la falta de chequeos en "`compilación"') es otro punto a destacar. Le saca rigurosidad al lenguaje incrementando su flexibilidad. Si se desea invocar a un método de un objeto y éste no existe, se buscará en la cadena de prototipo hasta terminar con la cadena, y en caso de que así sea, habrá un error en ejecución.

Una característica que también merece una mención es la de los módulos de ES6, ya que de una manera clara y concisa se pueden separar espacios de nombres y manejar las dependencias sin la necesidad de pensar en ellas.
