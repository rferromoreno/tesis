\section{Scope léxico}
\label{sec:scopelexico}

Existen dos modelos predominantes cuando hablamos de scope: El léxico y el dinámico. JavaScript hace uso del modelo de scope léxico, pero ?`qué significa exactamente esto?

El scope léxico, también llamado estático, hace la definición de los nombres durante la declaración, es decir, cuando estamos "`escribiendo código"'. En cambio, en el scope dinámico, los nombres se definen en ejecución dependiendo el contexto de la llamada. Analicemos el siguiente código:

\begin{lstlisting}
function foo() {
	console.log(a);
}

function bar() {
	var a = 3;
	foo();
}

var a = 2;

bar();
\end{lstlisting}

Como tenemos scope léxico, entonces la línea 2 imprimirá el valor 2, dado que \code{a} hace referencia al contexto léxico donde fue declarado (en este caso, \code{a} hace referencia al definido en la línea 10). 

Si tuvieramos scope dinámico, en cambio, se impriría el valor 3, dado que éste mecanismo resolvería la referencia revisando la pila de llamadas, haciendo referencia al valor de \code{a} asignado en la línea 6.