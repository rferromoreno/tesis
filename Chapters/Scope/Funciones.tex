\section{Scope por funciones}
\label{sec:scopefunciones}

La forma de crear un scope local en JavaScript es mediante funciones. Al momento de declarar una función, estamos "`creando"' un nuevo ámbito para la función. Dado el siguiente código:

\begin{lstlisting}
var a = "primera"

function uno() {
  var a = "segunda";

  function dos() {
    var a = "tercera";
    console.log(a);
  }

  dos();

  console.log(a);
}

uno();

console.log(a);
\end{lstlisting}

Existen tres scopes: 
\begin{itemize} 
\item El global, que es donde residen la variable \code{a} (con valor \code{``primera''}) y la función \code{uno}.
\item El scope dentro de la función \code{uno}, donde residen la variable \code{a} (con valor \code{``segunda''}) y la función \code{dos}.
\item El scope dentro de la función \code{dos}, donde reside la variable \code{a} (con valor \code{``tercera''}).
\end{itemize}

Dado que estamos hablando de "`distintas variables"' bajo el identificador \code{a}, el código presentado imprimirá por pantalla "`tercera"', "`segunda"' y "`primera"', en dicho orden.

Así como la declaración normal de una función sirve para "`generar un nuevo scope"', si la función no es necesaria, se puede hacer uso de un IIFE para omitir "`manchar"' el espacio de nombres.