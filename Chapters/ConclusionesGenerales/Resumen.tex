\section{Resumen}

JavaScript es expresivo y a su vez es sintácticamente agradable. Tiene facilidad para la escritura de programas, pero habiendo tantas combinaciones entre los constructores, la facilidad de lectura dependerá de la experiencia del programador. De alguna forma también es conciso en cuanto a la cantidad de líneas de código que se necesitan para hacer un programa.

Con un sistema de tipos extremadamente flexible, la seguridad en su sistema de tipos se ve reducida, ya que la coerción en ciertas expresiones pueden llevar a resultados inesperados. En este sentido, la confiabilidad del lenguaje se ve directamente afectada. 

Está claro que JavaScript es un lenguaje de "`scripting"' pero a su vez es multi-paradigma. Con un esfuerzo mediano, se aproxima a la idea del paradigma funcional tanto como a la del paradigma de orientación a objetos. Si bien es imposible que alcance la pureza en ambos de forma simultánea, la cobertura que le da a ambos paradigmas es aceptable.

Uno de los mayores problemas del lenguaje es su nombre. Hacer creer a los programadores que por llamarse JavaScript tendrá similitudes con Java es totalmente errado. De hecho, como se presentó en el documento, la semántica de ambos lenguajes están lejos una de otra.

El lenguaje tiene varios \textit{bugs}, es cierto, y eso no se puede negar. Teniéndole un poco de piedad al lenguaje, se puede decir que el mismo se "`creó en 10 días"', sin embargo tardó 20 años en agregar características que realmente eran necesarias para los programadores. 

No podemos afirmar que JavaScript es un lenguaje popular por sus características técnicas. Más bien, la popularidad probablemente se haya obtenido por factores externos, como la voluntad de su comunidad, o cuestiones empresariales o de mercado. Sin embargo, el gran acierto de JavaScript fue haber dado el "`primer golpe"' en la rama de la programación web. Cuesta imaginarse qué hubiera sido del lenguaje si éste no hubiera sido el pionero.

Lo esperanzador para el lenguaje es que después del lanzamiento de Node en 2009, más su actualización en la especificación en el año 2015, la comunidad creció de manera exponencial. No solo eso, se alienta a los programadores para que ellos mismos sean los que suban propuestas del lenguaje. En los últimos cuatro años, el estandar de ECMAScript se ha actualizado en Junio cada año. En este sentido, el lenguaje pareciera que seguirá evolucionando.

Para concluir, es necesario llamar a la reflexión del lector: \textit{El lenguaje es sólamente una herramienta}. Está en cáda uno de qué forma y para qué utiliza dicha "`herramienta"'.