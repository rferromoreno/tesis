\section{A futuro}

Las características del lenguaje vistas en éste documento solo forman parte de un subconjunto del lenguaje. De hecho, existen muchísimos otros aspectos para analizar del lenguaje. A continuación se le presentan al lector algunos conceptos que son dignos de destacar, para el caso de que quiera estudiar otras áreas del lenguaje:

\begin{itemize}
\item \keyword{Nuevas características sintácticas de ES6}: A partir de las nuevas versiones del estándar se introdujeron nuevos conceptos que facilitan la tarea del programador. Template literals, Spread y Rest operators, Default values, Destructuring
\item \keyword{Otros conceptos de ES6}: No solo se introdujeron características sintácticas en ES6, sino que aparecieron clases que son de ayuda para la \textit{meta-programación}, como por ejemplo: Symbol, Proxy, Iterators, Generators.
\item \keyword{Concurrencia}: JavaScript es un lenguaje \textit{single thread} y su fortaleza yace en el \textit{event loop}, del cual no se ha hablado en éste documento, pero es un concepto clave para entender cómo ese \textit{único hilo} maneja la concurrencia.
\item \keyword{Asincronía}: Directamente relacionado con el ítem anterior, se puede investigar sobre cómo se maneja la asincronía en JavaScript. Algunos conceptos destacables son: Callbacks, Promises, Observables, Generators, \code{await} y \code{async}.
\end{itemize}